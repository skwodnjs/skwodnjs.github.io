% ---------- This is the blog post preset. -----------

\documentclass[11pt,reqno]{amsart}

% PACKAGE --------------------------------------------

\usepackage{amsmath}
\usepackage{amssymb}
\usepackage{amsthm}

\usepackage{enumerate}
\usepackage[notcite,notref]{showkeys}
\usepackage[usenames]{color}
\usepackage{url}
\usepackage{hyperref}
\usepackage{kotex}

\usepackage{graphicx}

% PAGE SETTING ---------------------------------------

\textheight 22.5  true cm
\textwidth 15 true cm
\voffset -1.0 true cm
\hoffset -1.0 true cm
\marginparwidth= 2 true cm
\renewcommand{\baselinestretch}{1.2}

% blog post setting
\usepackage{titlesec}
\setlength{\parindent}{0pt}
\titleformat{\section}[block]{\normalfont\Large\bfseries}{}{0pt}{}
\titleformat{\subsection}[block]{\normalfont\large\bfseries}{}{0pt}{}
\titlespacing*{\subsection}{0pt}{3.0ex plus 1ex minus .2ex}{2.0ex plus .2ex}
\titleformat{\paragraph}[block]{\normalfont}{}{0pt}{}
\titlespacing*{\paragraph}{0pt}{3.0ex plus 1ex minus .2ex}{2.0ex plus .2ex}
% blog post setting end

% COSTOM COMMAND -------------------------------------

\renewcommand{\(}{\left(}
\renewcommand{\)}{\right)}
\renewcommand{\[}{\left[}
\renewcommand{\]}{\right]}

\newcommand{\inp}[2]{\langle #1,#2 \rangle}

\newcommand{\diag}{{\rm diag}}
\newcommand{\supp}{\text{supp }}

\newcommand{\R}{\mathbb{R}}
\newcommand{\Rp}{\mathbb{R}_+}
\newcommand{\Rpp}{\mathbb{R}_{++}}
\newcommand{\C}{\mathbb{C}}
\newcommand{\N}{\mathbb{N}}
\newcommand{\Z}{\mathbb{Z}}
\newcommand{\Q}{\mathbb{Q}}

\newcommand{\ep}{\epsilon}
\newcommand{\pa}{\partial}

\newcommand{\mcC}{\mathcal{C}}
\newcommand{\mcH}{\mathcal{H}}
\newcommand{\mcT}{\mathcal{T}}
\newcommand{\mcV}{\mathcal{V}}
\newcommand{\mcG}{\mathcal{G}}
\newcommand{\mcE}{\mathcal{E}}
\newcommand{\mcW}{\mathcal{W}}

\newcommand{\st}{ \; \big| \; }

% blog post setting
\usepackage{tcolorbox}
\newenvironment{textbox}
  {\begin{tcolorbox}[
    colback=gray!10, 
    colframe=gray!50, 
    boxrule=0.5pt,
    fontupper=\normalfont
  ]}
  {\end{tcolorbox}}
\newcommand{\subheading}[1]{\vspace{1em}{\noindent\large\bfseries \textlangle{} #1 \textrangle{} \par}\vspace{1em}}
% blog post setting end

% COSTOM COMMAND -------------------------------------

\newcommand{\M}{\mathfrak{M}}

% DOCUMENT -------------------------------------------

\begin{document}

\title[]{실해석학 2단원 벼락치기}
\author{Jeawon Na}
\date{2025. 6. 15.}
%\address{}
%\email{naa3000@skku.edu}

\maketitle

\vspace{0.5em}

교재: Walter Rudin - Real and Complex Analysis-McGraw-Hill Education (1986) \\
범위: CHAPER TWO,

\section{Vector Spaces}

\subsection{2.1 Definition}

Vector Space에 대해 정의함: addition과 scalar multiplication이 잘 정의된 space.

\subsection{2.2 Integration as a Linear Funtional}

Linear functional: scalar 을 return 하는 linear mapping. \\
$f \in C$ 에서, $\Lambda : C_c (X) \to C$가 $\Lambda f \ge 0$ if $f \ge 0$ 이면 positive linear functional이라 한다. \\
이때 $f \in C$이므로, $f < \infty$이다.

\section{Topological Preliminaries}

\subsection{2.3 Definition}

Topology의 기본 개념을 정의함.

\subsection{2.4 Theorem}

Suppose $K$ is compact and $F$ is closed, in a topological space $X$. If $F \subset K$, then $F$ is compact.

\paragraph{Corollary}

If $A \subset B$ and if $B$ has compact closure, so does $A$.

\subsection{2.5 Theorem}

Suppose $X$ is a Hausdorff space, $K \subset X$, $K$ is compact, and $p \in K^c$. Then there are open sets $U$ and $W$
such that $p \in U$, $K \subset W$, and $U \cap W = \emptyset$.

Hausdorff space는 두 point를 겹치지 않는 open set으로 분리시킬 수 있는 space로 정의되는데, compact와 점을 분리해낼 수 있다는 성질이
있다.

\paragraph{Corollary}

(a) Compact subsets of Hausdorff spaces are closed.
(b) If $F$ is closed and $K$ is compact in a Hausdorff space, then $F \cap K$ is compact.

Theorem 2.4에서는 closed set $F$가 $K$에 포함되어 있어야 했는데, $X$가 Hausdorff 라면, $K$가 $F$에 들어갈 필요가 없다.
$K$는 (a)에 의해 closed 이니까 $F \cap K$도 closed이고 Theorem 2.4에 의해 compact 안에 있는 closed는 compact이므로 (b)가 성립한다.

\subsection{2.6 Theorem}

If $\{ K_\alpha \}$ is a collection of compact subsets of a Hausdorff space and if $\bigcap _\alpha K _\alpha = \emptyset$, 
then some finite subcolleciton of $\{ K_\alpha \} $ also has empty intersection.

compact subset들의 교집합이 공집합이면 범인은 finite 개다.

\subsection{2.7 Theorem}

Suppose $U$ is open in a locally compact Hausdorff space $X$, $K \subset U$, and $K$ is compact. Then there is an open set $V$ 
with compact closure such that
\begin{equation}
  K \subset V \subset \bar{V} \subset U.
\end{equation}

locally compact Hausdorff space이면, open 과 compact 사이에 $V$가 들어갈 충분한 공간이 있다는 정리. 모든 점에 대해 compact closure를 가지는
neighborhood를 찾을 수가 있고, Hausdorff 덕분에 neighborhood 각각을 $U$ 안으로 밀어 넣을 수 있다. ($U^c$의 모든 점과 $K$를 떼어놓는 open set을
이용해서.)

\subsection{2.8 Definition}

Let $f$ be a real (or extended-real) function on a fopological space. If 
\begin{equation}
  \{ x : f (x) > \alpha \}
\end{equation}
is open for every real $\alpha$, $f$ is said to be lower semicontinuous. If 
\begin{equation}
  \{ x : f (x) < \alpha \}
\end{equation}
is open for every real $\alpha$, $f$ is said to be upper semicontinuous.

(a) Characteristic functions of open sets are lower semicontinuous.
(b) Characteristic functions of closed sets are upper semicontinuous.

$f$가 continuous이면 $f^{-1} (V)$ is open for every $V \subset Y$ 인데, 이는 $f^{-1} ((\alpha, \beta))$ is open for every real $\alpha, 
\beta$랑 같은 뜻이고, $\{ x : \alpha< f (x) < \beta \}$랑 같은 뜻이다. 이렇게 놓고 보면, upper semicontinuous랑 lower semicontinuous는 
자연스러운 정의로 보인다.

생각해보면 Characteristic funciton은 0 또는 1의 값을 가지므로, upper semicontinuous 이려면 0일 때 open이어야 됨. 즉, closed sets의 characteristic
functions는 upper semicontinuous. 반대로 lower semicontinuous 이려면 1일 때 open이어야 됨. $(\alpha, \infty]$의 preimage가 open이어야 하니까. 
그래서 (a) 성립.

(c) The supremum of any collection of lower semicontinuous funcitons is lower semicontinuous. 
The infimum of any collection of upper semicontinuous functions is upeer semicontinuous.

\subsection{2.9 Definition}

The support of a complex function $f$ on a topological space $X$ is the closure of the set 
\begin{equation}
  \{ x : f (x) \neq 0 \}.
\end{equation}

The collection of all continuous complex fucntions on $X$ whose support is compact is denoted by $C_c(X)$.

\subsection{2.10 Theorem}

Let $X$ and $Y$ be topological spaces, and let $f : X \to Y$ be continuous. If $K$ is a compact subset of $X$, then $f (K)$ is compact.

\paragraph{Corollary}

The range of any $f \in C_c(X)$ is a compact subset of the complex plane.

\subsection{2.11 Notation}

In this chapter the following conventions will be used. The notation
\begin{equation}
  K \prec f
\end{equation}
will mean that $K$ is a compact subset of $X$, that $f \in C_c(X)$, that $0 \le f(x) \le 1$ for all $x \in X$, and that $f(x) = 1$ for all $x \in K$.
The notation 
\begin{equation}
  f \prec V
\end{equation}
will mean that $V$ is open, that $f \in C_c(X)$, $0 \le f \le 1$, and that the support of $f$ lies in $V$. The notation
\begin{equation}
  K \prec f \prec V
\end{equation}
will be used to indicate that both (-) and (-) holds.

\subsection{2.12 Urysohn's Lemma}

Suppose $X$ is locally compact Hausdorff space, $V$ is open in $X$, $K \subset V$, and $K$ is compact. Then there exists an $f \in C_c(X)$, such that 
\begin{equation}
  K \prec f \prec V.
\end{equation}

\subsection{2.13 Theorem}

Suppose $V_1, \dots, V_n$ are open subsets of a locally compact Hausdorff space $X$, $K$ is compact, and
\begin{equation}
  K \subset V_1 \cup \cdots \cup V_n.
\end{equation}
Then there exist functions $h_i \prec V_i (i = 1, \dots, n)$ such that 
\begin{equation}
  h_1 (x) + \cdots + h_n (x) = 1 \qquad (x \in K).
\end{equation}

$h = h_1 + \cdots + h_n$이라 하면, $0 \le h \le 1$이고, $h \in C_c(X)$이고, (왜냐면 $C_c(X)$는 vector space, 더하기에 대해 닫혀있음), $h(x) = 1$ for every 
$x \in K$이까 $K \prec h$라 할 수 있겠다. (맞을걸?)

\section{The Riesz Representation Theorem}

\subsection{2.14 Theorem}

Let $X$ be a locally compact Hausdorff space, and let $\Lambda$ be a positive linear funcitonal on $C_c (X)$. Then there exists a $\sigma-algebra \M$
in $X$ which contains all Borel sets in $X$, and there exists a unique positive measure $\mu$ on $\M$ which represents $\Lambda$ in the sense that
(a) $\Lambda f = \int_X f d \mu$ for every $f \in C_c(X)$, and which has the following additional properties:
(b) $\mu (K) < \infty$ for every compact set $K \subset X$.
(c) For every $E \in \M$, we have \begin{equation}
  \mu(E) = \inf \{ \mu (V): E \subset V, V \text{ open} \}.
\end{equation}
(d) The relation \begin{equation}
  \mu(E) = \sup \{ \mu (K) : K \subset E, K \text{ compact} \}
\end{equation}
holds for every open set $E$, and for every $E \in \M$ with $\mu (E) < \infty$.
(e) If $E \in \M$, $A \subset E$, and $\mu(E) = 0$, then $A \in \M$.

어떤 positive linear functional 이 주어졌을 때, 이걸 가지고 unique한 positive measure를 하나 찾아낼 수 있다는 의미이고, 그 positive measure는 여러 좋은 성질들을 
가지고 있음. 근데 모든 positive measure가 이런 좋은 성질들을 만족하는건 아님.

positive linear functional이 주어졌을 때, (a) ~ (e)의 모든 조건을 만족하는 positive measure는 unique하다는걸 우선은 알 수 있다. 
이걸 먼저 증명하고 나면, 존재성에 대한 증명만 하면 된다.

\paragraph{Construction of $\mu$ and $\M$}

For every open set $V$ in $X$, definte 
\begin{equation}
  \mu (V) = \sup \{ \Lambda f : f \prec V \}.
\end{equation}

For every $E \subset X$, 
\begin{equation}
  \mu (E) = \inf \{ \mu (V) : E \subset V, V \text{ open} \}.
\end{equation}

Let $\M_F$ be the class of all $E \subset X$ which satisfy two conditinos: $\mu(E) < \infty$, and 
\begin{equation}
  \mu (E) = \sup \{ \mu (K) : K \subset E, K \text{ compact} \}.
\end{equation}
Finally, let $\M$ be the class of all $E \subset X$ such that $E \cap K \in \M_F$ for every compact $K$.

Riesz 표현 정리의 증명은 이렇게 설계한 $\mu$와 $\M$이 (a) ~ (e)를 만족하는 measure인지 확인하는 것으로 진행된다.
참고로, $\M_F$는 $\M$ 중에서 $\mu(E) < \infty$인 집합들의 collection과 같다. 즉, 이렇게 설계하면 $\mu(E) < \infty$인
$E$는 조건 $\mu (E) = \sup \{ \mu (K) : K \subset E, K \text{ compact} \}$을 만족한다는 것이다. 증명과정에서 확인할 
수 있다.

uniqueness에 대한 증명을 했다면, 이렇게 설계한 $\mu$ 밖에 없다 라고 이야기할 수 있겠다.

\paragraph{STEP I}

If $E_1, E_2, E_3, \dots$ are arbitrary subsets of $X$, then
\begin{equation}
  \mu \( \bigcup_{i =1} ^\infty \) \le \sum_{i=1}^\infty \mu(E_i).
\end{equation}

아직 $\mu$가 measure는 아니지만, 위 식은 measure가 만족해야 할 성질 중 하나이다. 겹치는 부분이 있을 수도 있는 집합들에 대해서,
집합들을 합쳐놓고 그 크기를 측정하면 각각의 크기를 측정한 뒤에 합한 것보다 작아질 수 있다. 만약, disjoint한 집합들에 대해서 반대방향의
부등식이 성립한다면, countably additive가 성립한다고 할 수 있겠다.

\paragraph{STEP II}

If $K$ is compact, then $K \in \M_F$ and
\begin{equation}
  \mu(K) = \inf \{ \Lambda f : K \prec f \}.
\end{equation}
This implies assertion (b) of the theorem.

일단 compact set $K$는 $\M_F$에 속한다. 근데 일단 $\M_F$에 속하려면 첫 번째 조건으로 $\mu (K) < \infty$이어야 하므로, 이걸 
증명한다는 것은 theorem (b) 부분을 보이는 것과 같다. 그리고 위 식에서, $K$ 위에서 모두 1의 값을 가지는 함수 $f$들의 infimum으로 
$\mu(K)$를 계산할 수 있다는 것은 함수 $f$를 통해 위에서부터 누르는 것으로 $\mu(K)$를 계산해낼 수 있다는 것을 의미한다.

지금은 compact set $K$가 $\M_F$에 속한다는 것을 보였는데, 다음에는 open set $V$가 $\M_F$에 속하는지를 살펴본다.

\paragraph{STEP III}

Every open set satisfies (3). Hence $\M_F$ conmtains every open set $V$ with $\mu(V) < \infty$.

(3)번 식은 (지금 tex에서는 하나도 맞지 않겠지만) $\M_F$의 두 번째 조건이다.

\paragraph{STEP IV}

Suppose $E = \bigcup _{i=1}^\infty E _i$, where $E_1, E_2, E_3, \dots$ are pairwise disjoint members of $\M_F$. Then
\begin{equation}
  \mu(E) = \sum _{i=1}^\infty \mu (E_i).
\end{equation}
If, in addition, $\mu (E) < \infty$, then also $E \in \M_F$.

$\M_F$에 대해서 countably additivity가 (어느정도) 성립함을 보인다. 왜 '어느정도'냐면 무한대가 되어버리면 $\M_F$에 속하지 않으니까
닫혀있지가 않다. 이건 나중에 $\M$으로 확장시키면서 해결할 수 있다.

\paragraph{STEP V}

If $E \in \M_F$ and $\ep > 0$, there is a compact $K$ and an open $V$ such that $K \subset E \subset V$ and $\mu(V - K) < \ep$.

일종의 보조정리 느낌.

\paragraph{STEP VI}

If $A \in \M_F$ and $B \in \M_F$, then $A - B$, $A \cup B$, $A \cap B$ belong to $\M_F$.

일종의 보조정리 느낌. $\M_F$를 $\M$으로 확장하기 위한 준비과정.

\paragraph{STEP VII}

$\M$ is a $\sigma$-algebra in $X$ which contains all Borel sets.

\paragraph{STEP VIII}

$\M_F$ constains of precisely those sets $E \in \M$ for shich $\mu(E) < \infty$.

\paragraph{STEP IX}

$\mu$ is a measure on $\M$.

\paragraph{STEP X}

For every $f \in C_c (X)$, $\Lambda f = \int _X f d\mu$.

\section{Regularity Properties of Borel Measures}

\subsection{2.15 Definition}

A measure $\mu$ defined on the $\sigma$-algebra of all Borel sets in a locally compact Hausdorff space $X$ is called a Borell measure
on $X$. If $\mu$ is positive, a Borel set $E \subset X$ is outer regular or inner regular, respectively, if $E$ has property (c) or (d)
of Theorem 2.14. If every Borel set in $X$ is both outer and inner regular, $\mu$ is called regular.

일반적인 Borel measure는 open set $V$에 의해 outer로 measure를 구할 수 없다. 근데 Theorem 2.14에 의해서, positive linear functional 에 의해 
만들어진 $\mu$는 모든 Borel set을 $outer regular$로 만든다. 그리고 이 $\mu$는 최소한 finite measure를 갖는 $E \subset X$에 대해서는 inner 
regular 이지만, finite가 아닌 measure에 대해서는 inner regular이 아닐 수도 있다. exercise 17에 그 아닌 예시가 나와있다. 만약에, finite가 아닌
measure에 대해서도 inner regular이 된다면, 그 때 $\mu$는 regular이라고 부른다.

\subsection{2.16 Definition}

A set $E$ in a topological space is called $\sigma$-compact if $E$ is a countable union of compact sets.

$\sigma$가 countable을 의미한다. $\Sigma$가 countable sum을 의미하는거처럼.

A set $E$ in a measrue space (with measure $\mu$) is said to have $\sigma$-finite measure if $E$ is a countable union of sets $E_i$ with 
$\mu(E_i) < \infty$.

Theorem 2.14에서 construct 한 $\mu$에 대해서, 모든 $\sigma$-compact set은 $\sigma$-finite measure를 가진다. 왜냐면 모든 compact set은 finite 
measure을 가지니까. 또한 이때, $E \in \M$ and $E$ has $\sigma$-finite measure이면, $E$ is inner regular이다 (왜?).

\subsection{2.17 Theorem}

Suppose $X$ is a locally compact, $\sigma$-compact Hausdorff space. If $\M$ and $\mu$ are as described in the statement of Theorem 
2.14, then $\M$ and $\mu$ have the following properties:
(a) If $E \in \M$ and $\ep > 0$, there is a closed set $F$ and an open set $V$ such that $F \subset E \subset V$ and $\mu(V - F) < \ep$.
(b) $\mu$ is a regular Borel measure on $X$.
(c) If $E \in \M$, there are sets $A$ and $B$ such that $A$ is an $F_\sigma$, $B$ is a $G_\delta$, $A \subset E \subset B$, and $\mu(B - A) = 0$.

Theorem 2.14에서 $X$가 $\sigma$-compact 하다는 조건만 추가되면, 위와 같은 특징이 추가된다는 뜻이다.
원래 $\M_F$에 속하는 $E$에 대해서, compact한 $K$와 open $V$를 찾을 수 있었다. 근데 (a)에 의하면, $\M$에 속하는 $E$에 대해서도, 찾을 수 있기는 한데 
compact까지는 아니고 대신 closed인 $F$를 찾을 수 있다는 뜻이다. $\mu (E) < \infty$이면 그중에서도 compact한, 더 좋은 집합을 찾을 수 있긴 한데, (a)는 
$\mu (E) < \infty$가 아니어도 closed 인 $F$를 찾을 수 있게 해준다.
(b)는 원래 open set이거나 finite measure를 가지는 set에 대해서만 성립하는 특징이었는데, 이젠 모든 Borel set에 대해서 inner regular도 만족하게 된다.
$F_\sigma$는 countable union of closed sets이고, $G_\delta$는 countable intersection of open sets이다. (c)는 $E$를 사이에 두는 엄청 가까운(?) 
$A$와 $B$를 찾아낼 수 있다는 뜻이다.

\subsection{2.18 Theorem}

Let $X$ be a locally compact Hausdorff space in which every open set is $\sigma$-compact. Let $\lambda$ be any positive Borel measure on $X$ 
such that $\lambda (K) < \infty$ for every compact set $K$. Then $\lambda$ is regular.

이번에는 모든 open set이 $\sigma$-compact인 조건이 추가된다. 앞선 조건은 전체 $X$가 $\sigma$-compact일 때 추가되는 조건을 소개했고, 지금은 각 open set들이 
$sigma$-compact일 때 추가되는 조건에 대해 소개한다. 참고로, $\R^n$은 전체가 $\sigma$-compact이면서 각 open set들도 $\sigma$-compact이다.

\section{Lebesgue Measure}

\subsection{2.19 Euclidean Spaces}

Euclidean $k$-dimensional space $\R^k$ 에 대한 정의, vector space이면서 norm 이 정의되고 metric도 자연스럽게 정의되는 metric space.

If $E \subset \R^k$ and $x \in \R^k$, the translate of $E$ by $x$ is the set 
\begin{equation}
  E + x = \{ y + x : y \in E \}.
\end{equation}

A set of the form 
\begin{equation}
  W = \{ x : \alpha _i < \xi_i < \beta _i , 1 \le i \le k \},
\end{equation}
or any set obtained by replacing any or all of the $<$ signs in (4) by $\le$, is called a $k$-cell; its volumne is defined to be
\begin{equation}
  \rm vol (W) = \prod_{i=1}^k (\beta_i - \alpha_i)
\end{equation}

If $a \in \R^k$ and $\delta > 0$, we shall call the set 
\begin{equation}
  Q(a; \delta) = \{ x : \alpha_i \le \xi_i < \alpha_i + \delta, 1 \le i < k \}
\end{equation}
the $\delta$-box with corner at $a$. Here $a = (\alpha_1, \dots, \alpha_k)$.

For $n = 1, 2, 3, \dots,$ we let $P_n$ be the set of all $x \in \R^k$ whose coordinates are integral multiples of $2^{-n}$, 
and we let $\Omega_n$ be the collection of all $2^{-n}$ boxes with corners at points of $P_n$. We shall need the following four properties of 
$\{ \Omega_n \}$. The first therr are obvious by inspection.

$\{ \Omega_n \}$ 은 $2^{-n}$ 간격으로 격자를 다 쪼개놓은 공간이라 생각하면 된다. $P_n$이 그 격자이고, $\Omega_n$들이 그 boxes라고 생각하면 된다.

(a) If $n$ is fixed, each $x \in \R^k$ lies in one and only one member of $\Omega_n$.
(b) If $Q' \in \Omega_n$, $Q'' \in \Omega_r$, and $r < n$, then either $Q' \subset Q''$ or $q' \cap Q'' = \emptyset$.
(c) If $Q \in \Omega_r$, then $rm vol(Q) = 2^{-rk}$; and if $n > r$, the set $P_n$ has exactly $2 ^ {(n-r)k}$ points in $Q$.
(d) Every nonempty open set in $\R^k$ is a countable union of disjoint boxes belonging to $\Omega_1 \cup \Omega_2 \cup \Omega_3 \cup \cdots$.

(a), (b), (c)는 box를 그려보면 금방 알 수 있다. (d)는 box들을 basis로 생각할 수 있다는 느낌으로 볼 수 있겠다.

\subsection{2.20 Theorem}

There exists a positive complete measure $m$ defined on a $\sigma$-algebra $\M$ in $\R^k$, with the following properties:
(a) $m(W) = \rm vol(W)$ for every $k$-cell $W$.
(b) $\M$ contains all Borel sets in $\R^k$; more precisely, $E \in \M$ if and only if there are sets $A$ and $B \in \R^k$ such that 
$A \subset E \subset B$, $A$ is an $F_\sigma$, $B$ is a $G_\delta$, and $m (B - A) = 0$. Also, $M$ is regular.
(c) $m$ is translation-invariant, i.e., 
\begin{equation}
  m(E + x) = m(E)
\end{equation}
for every $E \in \M$ and every $x \in \R^k$.
(d) If $\mu$ is any positive translation-invariant Borel measure on $\R^k$ such that $\mu(K) < \infty$ for every compact set $K$, 
then there is a constant $c$ such that $\mu (E) = c m (E)$ for all Borel sets $E \subset \R^k$.
(e) To every linear transformation $T$ of $\R^k$ into $\R^k$ corresponds a real number $\Delta (T)$ such that 
\begin{equation}
  m(T(E)) = \Delta (T) m (E) 
\end{equation}
for every $E \in \M$. In particular, $m(T(E)) = m(E)$ when $T$ is a rotation.

$\R^k$ 위에, 적당한 (그리고 굉장히 직관적인) $\M$과 $m$을 정의할 수 있다는 뜻이다. (b)에서, $X$가 $\sigma$-compact 하다면 리즈 표현 정리에 의해
positive linear functional로 만든 measure가 해당 사실을 만족한다는 것을 증명한 정기 있다. 아마도, 이 Theorem을 정의할 때, 적당한 positive linaer
functional 을 주고, 그걸로 measure를 정의하면 되지 않을까 생각해볼 수 있다.

(d)는, translation-invariant한 또 다른 measure $\mu$가 있다면, 그것과의 관계에 대한 설명이다.
(e)는, 치환적분과 비슷한 느낌일 것 같다.

The members of $\M$을 Lebesgue measurable sets in $\R^k$, $m$을 Lebesgue measure on $\R^k$라 한다.

\subsection{2.21 Remarks}

만약 $m$ 이 Lebesgue measure on $\R^k$이면, 관례적으로 $L^1 (m)$ 대신 $L^1 (\R^k)$이라 쓰기도 한다. 등등... 이따 함 읽어보는게 좋을듯?

\subsection{2.22 Theorem}

If $A \subset \R^1$ and every subset of $A$ is Lebesgue measurable then $m(A) = 0$.

\paragraph{Corollary}

Every set of positive measure has nonmeasurable subsets.

\subsection{2.23 Determinants}

The scale factors $\Delta(T)$ that occur in Theorem 2.20(e) can be interpreted algebraically by means of determinants.

\section{Continuity Properties of Measurable Functions}

\subsection{2.24 Lusin's Theorem}

Suppose $f$ is a complex measurable function on $X$, $\mu(A) < \infty$, $f(x) = 0$ if $x \notin A$, and $\ep > 0$. Then there exists a $g \in C_c(X)$
such that \begin{equation}
  \mu(\{x : f(x) \neq g(x) \}) < \ep.
\end{equation}
Furthermore, we may arrange it so that \begin{equation}
  \sup_{x \in X} |g(x)| \le \sup_{x \in X} |f(x)|.
\end{equation}

\paragraph{Corollary}

Assume that the hypotheses of Lusin's theorem are satisfied and that $|f| \le 1$. Then there is a sequence $\{g_n\}$ such that 
$g_n \in C_c(X)$, $|g_n| \le 1$, and \begin{equation}
  f(x) = \lim _{n \to \infty} g_n (x) \qquad \text{a.e.}
\end{equation}

\subsection{2.25 The Vitali-Caratheodory Theorem}

Suppose $f \in L^(\mu)$, $f$ is real-valued, and $\ep > 0$. Then there exist functions $u$ and $v$ on $X$ such that $u \le f \le v$,
$u$ is upper semicontinuous and bounded above, $v$ is lower semicontinuous and bounded below, and \begin{equation}
  \int_X (v - u) d\mu < \ep.
\end{equation}

\end{document}
