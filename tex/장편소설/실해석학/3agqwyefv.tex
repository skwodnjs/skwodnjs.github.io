% ---------- This is the blog post preset. -----------

\documentclass[11pt,reqno]{amsart}

% PACKAGE --------------------------------------------

\usepackage{amsmath}
\usepackage{amssymb}
\usepackage{amsthm}

\usepackage{enumerate}
\usepackage[notcite,notref]{showkeys}
\usepackage[usenames]{color}
\usepackage{url} 
\usepackage{kotex}

\usepackage{graphicx}

% PAGE SETTING ---------------------------------------

\textheight 22.5  true cm
\textwidth 15 true cm
\voffset -1.0 true cm
\hoffset -1.0 true cm
\marginparwidth= 2 true cm
\renewcommand{\baselinestretch}{1.2}

% blog post setting
\usepackage{titlesec}
\setlength{\parindent}{0pt}
\titleformat{\section}[block]{\normalfont\Large\bfseries}{}{0pt}{}
\titleformat{\subsection}[block]{\normalfont\large\bfseries}{}{0pt}{}
\titlespacing*{\subsection}{0pt}{3.0ex plus 1ex minus .2ex}{2.0ex plus .2ex}
\titleformat{\paragraph}[block]{\normalfont}{}{0pt}{}
\titlespacing*{\paragraph}{0pt}{3.0ex plus 1ex minus .2ex}{2.0ex plus .2ex}
\setcounter{equation}{9}
% blog post setting end

% COSTOM COMMAND -------------------------------------

\renewcommand{\(}{\left(}
\renewcommand{\)}{\right)}
\renewcommand{\[}{\left[}
\renewcommand{\]}{\right]}

\newcommand{\inp}[2]{\langle #1,#2 \rangle}

\newcommand{\diag}{{\rm diag}}
\newcommand{\supp}{\text{supp }}

\newcommand{\R}{\mathbb{R}}
\newcommand{\Rp}{\mathbb{R}_+}
\newcommand{\Rpp}{\mathbb{R}_{++}}
\newcommand{\C}{\mathbb{C}}
\newcommand{\N}{\mathbb{N}}
\newcommand{\Z}{\mathbb{Z}}
\newcommand{\Q}{\mathbb{Q}}

\newcommand{\ep}{\epsilon}
\newcommand{\pa}{\partial}

\newcommand{\mcC}{\mathcal{C}}
\newcommand{\mcH}{\mathcal{H}}
\newcommand{\mcT}{\mathcal{T}}
\newcommand{\mcV}{\mathcal{V}}
\newcommand{\mcG}{\mathcal{G}}
\newcommand{\mcE}{\mathcal{E}}
\newcommand{\mcW}{\mathcal{W}}

\newcommand{\st}{ \; \big| \; }

% blog post setting
\usepackage{tcolorbox}
\newenvironment{textbox}
  {\begin{tcolorbox}[
    colback=gray!10, 
    colframe=gray!50, 
    boxrule=0.5pt,
    fontupper=\normalfont
  ]}
  {\end{tcolorbox}}
\newcommand{\subheading}[1]{\vspace{1em}{\noindent\large\bfseries \textlangle{} #1 \textrangle{} \par}\vspace{1em}}
% blog post setting end

% COSTOM COMMAND -------------------------------------

\newcommand{\M}{\mathfrak{M}}

% DOCUMENT -------------------------------------------

\begin{document}

\title[]{실해석학 스터디 5주차}
\author{Jeawon Na}
\date{2025. 6. 3.}
%\address{}
%\email{naa3000@skku.edu}

\maketitle

\vspace{0.5em}

교재: Walter Rudin - Real and Complex Analysis-McGraw-Hill Education (1986) \\
범위: CHAPER TWO, "POSITIVE BOREL MEASURES": The Riesz Representation Theorem

\section{The Riesz Representation Theorem}

\subsection{2.14}

\begin{textbox}
  Let $X$ be a locally compact Hausdorff space, and let $\Lambda$ be a positive linear functional on $C_c(X)$. \\
  Then there exists a $\sigma$-algebra $\M$ in $X$ which contains all Borel sets in $X$, 
  and there exists a unique positive measure $\mu$ on $\M$ which represents $\Lambda$ in the sense that \\ \\
  (a) $\Lambda f = \int_X f d \mu$ for every $f \in C_c(X)$, and which has the following additional properties: \\
  (b) $\mu(K) < \infty$ for every compact set $K \subset X$. \\
  (c) For every $E \in \M$, we have 
  \begin{equation*}
    \mu(E) = \inf \{ \mu(V) : E \subset V, \, V \text{ open} \}.
  \end{equation*} \\
  (d) The relation
  \begin{equation*}
    \mu(E) = \sup \{ \mu(K) : K \subset E, \, K \text{ compact} \}
  \end{equation*}
  holds for every open set $E$, and for every $E \in \M$ with $\mu(E) < \infty$. \\
  (e) If $E \in \M$, $A \subset E$, and $\mu(E) = 0$, then $A \in \M$.
\end{textbox}

지난 시간에 STEP I부터 III까지를 증명했었다.\\ \\
STEP I은 임의의 집합들에 대해, 합집합을 먼저 한 다음에 $\mu$에 넣은 것보다 먼저 $\mu$에다가 넣고 
합집합을 한 것이 더 크다는 내용의 Theorem이었다. 아직 $\mu$가 measure는 아니지만, measure의 
관점에서 생각해 보면, 임의의 집합의 크기를 측정한 결과를 합쳤을 때 겹치는 부분이 중복 측정되기 때문에 
먼저 합집합한 뒤에 크기를 측정한 것보다 커질 거라는 사실을 직관적으로 생각해볼 수 있다. \\ \\
STEP 2는 compact set의 특징에 대해 알아보았던 Theorem이었다. compact하면 $\M_F$에 포함되고, 
$\Lambda f$에 의해 $\inf$로도 근사할 수 있다는 내용이었다. 그리고 $\mu(K)$가 finite하다는 사실도 
자연스럽게 증명할 수 있었다. \\ \\
STEP III은 open set에 대한 내용이었다. open set이면 compact set에 의해 $\inf$으로 $\mu$의 
값을 근사할 수 있다는 내용이었다. 그래서 finite한 $\mu$ 값을 가진 open set은 $\M_F$에 포함된다는 
사실을 확인했다. \\ \\

이제, 지난 증명에 이어서 진행해보자.



\end{document}

