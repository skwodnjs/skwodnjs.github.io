% ---------- This is the blog post preset. -----------

\documentclass[11pt,reqno]{amsart}

% PACKAGE --------------------------------------------

\usepackage{amsmath}
\usepackage{amssymb}
\usepackage{amsthm}

\usepackage{enumerate}
\usepackage[notcite,notref]{showkeys}
\usepackage[usenames]{color}
\usepackage{url} 
\usepackage{kotex}

\usepackage{graphicx}

% PAGE SETTING ---------------------------------------

\textheight 22.5  true cm
\textwidth 15 true cm
\voffset -1.0 true cm
\hoffset -1.0 true cm
\marginparwidth= 2 true cm
\renewcommand{\baselinestretch}{1.2}

% blog post setting
\usepackage{titlesec}
\setlength{\parindent}{0pt}
\titleformat{\section}[block]{\normalfont\Large\bfseries}{}{0pt}{}
\titleformat{\subsection}[block]{\normalfont\large\bfseries}{}{0pt}{}
\titlespacing*{\subsection}{0pt}{3.0ex plus 1ex minus .2ex}{2.0ex plus .2ex}
\titleformat{\paragraph}[block]{\normalfont}{}{0pt}{}
\titlespacing*{\paragraph}{0pt}{3.0ex plus 1ex minus .2ex}{2.0ex plus .2ex}
\setcounter{equation}{13}
% blog post setting end

% COSTOM COMMAND -------------------------------------

\renewcommand{\(}{\left(}
\renewcommand{\)}{\right)}
\renewcommand{\[}{\left[}
\renewcommand{\]}{\right]}

\newcommand{\inp}[2]{\langle #1,#2 \rangle}

\newcommand{\diag}{{\rm diag}}
\newcommand{\supp}{\text{supp }}

\newcommand{\R}{\mathbb{R}}
\newcommand{\Rp}{\mathbb{R}_+}
\newcommand{\Rpp}{\mathbb{R}_{++}}
\newcommand{\C}{\mathbb{C}}
\newcommand{\N}{\mathbb{N}}
\newcommand{\Z}{\mathbb{Z}}
\newcommand{\Q}{\mathbb{Q}}

\newcommand{\ep}{\epsilon}
\newcommand{\pa}{\partial}

\newcommand{\mcC}{\mathcal{C}}
\newcommand{\mcH}{\mathcal{H}}
\newcommand{\mcT}{\mathcal{T}}
\newcommand{\mcV}{\mathcal{V}}
\newcommand{\mcG}{\mathcal{G}}
\newcommand{\mcE}{\mathcal{E}}
\newcommand{\mcW}{\mathcal{W}}

\newcommand{\st}{ \; \big| \; }

% blog post setting
\usepackage{tcolorbox}
\newenvironment{textbox}
  {\begin{tcolorbox}[
    colback=gray!10, 
    colframe=gray!50, 
    boxrule=0.5pt,
    fontupper=\normalfont
  ]}
  {\end{tcolorbox}}
\newcommand{\subheading}[1]{\vspace{1em}{\noindent\large\bfseries \textlangle{} #1 \textrangle{} \par}\vspace{1em}}
% blog post setting end

% COSTOM COMMAND -------------------------------------

\newcommand{\M}{\mathfrak{M}}

% DOCUMENT -------------------------------------------

\begin{document}

\title[]{실해석학 스터디 5주차}
\author{Jeawon Na}
\date{2025. 6. 3.}
%\address{}
%\email{naa3000@skku.edu}

\maketitle

\vspace{0.5em}

교재: Walter Rudin - Real and Complex Analysis-McGraw-Hill Education (1986) \\
범위: CHAPER TWO, "POSITIVE BOREL MEASURES": The Riesz Representation Theorem

\section{The Riesz Representation Theorem}

\subsection{2.14}

\begin{textbox}
  Let $X$ be a locally compact Hausdorff space, and let $\Lambda$ be a positive linear functional on $C_c(X)$. \\
  Then there exists a $\sigma$-algebra $\M$ in $X$ which contains all Borel sets in $X$, 
  and there exists a unique positive measure $\mu$ on $\M$ which represents $\Lambda$ in the sense that \\ \\
  (a) $\Lambda f = \int_X f d \mu$ for every $f \in C_c(X)$, and which has the following additional properties: \\
  (b) $\mu(K) < \infty$ for every compact set $K \subset X$. \\
  (c) For every $E \in \M$, we have 
  \begin{equation*}
    \mu(E) = \inf \{ \mu(V) : E \subset V, \, V \text{ open} \}.
  \end{equation*} \\
  (d) The relation
  \begin{equation*}
    \mu(E) = \sup \{ \mu(K) : K \subset E, \, K \text{ compact} \}
  \end{equation*}
  holds for every open set $E$, and for every $E \in \M$ with $\mu(E) < \infty$. \\
  (e) If $E \in \M$, $A \subset E$, and $\mu(E) = 0$, then $A \in \M$.
\end{textbox}

지난 시간에는 $\mu$와 $\M$을 적당히 정의한 뒤, STEP I부터 IV까지 증명했었다. 이번에는 지난 증명에 이어서 진행해보자.

\paragraph{STEP V}

If $E \in \M_F$ and $\ep > 0$, there is a compact $K$ and an open $V$ such that $K \subset E \subset V$ 
and $\mu (V - K) < \ep$. \\ \\

어떤 집합 $E$가 $\M_F$에 있으면 $E$를 사이에 둔 상태에서 충분히 가까운 compact $K$와 open $V$를 찾을 수 있다.

\subheading{proof}

$\mu(E)$는 정의에 의해 $\mu(E) = \inf\{ \mu(V) : E \subset V, V \text{ open} \}$이고, $\M_F$에 속한다고 했으므로 
$\mu(E) = \sup\{ \mu(K) : K \subset E, K \text{ compact} \}$이다. 따라서, 다음과 같은 
$K \subset E \subset V$를 찾을 수 있다.

\begin{equation*}
  \mu(V) - \frac{\ep}{2} < \mu(E) < \mu(K) + \frac{\ep}{2}
\end{equation*}

이 때 locally compact Hausdorff space에서 compact이면 closed이므로 $K^c$는 open이고, 따라서 
$V - K = V \cap K^c$는 open이다. STEP III에 의해 open이면 $\M_F$에 속한다. 따라서 $V - K \in \M_F$이고, 
STEP IV에 의해
\begin{equation*}
  \mu (K) + \mu(V-K) = \mu(V) < \mu(K) + \ep
\end{equation*}
이다. (K와 V-K는 당연히 disjoint이다.) 따라서 $\mu(V-K) < \ep$.

\paragraph{STEP VI}

If $A \in \M_F$ and $B \in \M_F$, then $A-B$, $A\cup B$, and $A \cap B$ belong to $\M_F$. \\ \\

\subheading{proof}

Let $\ep > 0$. STEP  V에 의해, $A$와 $B$에 대해 $K_1 \subset A \subset V_1$, $K_2 \subset A \subset V_2$, 
and $\mu (V_1 - K_1)<\ep$, $\mu (V_2 - K_2)<\ep$인 $K_i$와 $V_i$를 찾을 수 있다. 이 때 간단한 집합 연산에 의해
\begin{equation*}
  A - B \subset V_1 - K_2 \subset (V_1 - K_1) \cup (K_1 - V_2) \cup (V_2 - K_2)
\end{equation*} 가 성립한다.
 * 직관적인 이해: 우변 $V_1 - K_1$에서, 빠질땐 $K_1$의 일부만 빠짐. (겹치는 부분만). 근데 합칠때는 $K_1 - V_2$에서 
 다 합침. 그리고 다시 $V_2$에서 일부만 빠짐. 만약에 합칠 때 빠졌던 부분(일부)만 그대로 합친다면 좌변과 동일. 근데 그거보다
 더 많이 합치니까, 커짐. \\ \\
따라서, 다음이 성립한다.
\begin{equation}
  \mu(A-B) \le \mu((V_1 - K_1) \cup (K_1 - V_2) \cup (V_2 - K_2)) \le \ep + \mu(K_1 - V_2) + \ep.
\end{equation}
$K_1 - V_2$는 compact subset of $A-B$이다. 왜냐하면 $K_1 - V_2 = K_1 \cap V_2^c$이고, Hausdorff space에서 
closed와 compact의 교집합은 compact이니까. (see Corollary (b) of Theorem 2.5). \\ \\
그러므로, $\sup \{ \mu (K) : K \subset A-B, K \text{ compact} \} < \mu(A-B)$일 수 없다. 왜냐면 $\mu(A-B)$보다
조금이라도 작은 값에 대해서는 그것보다 큰 $K_1, V_2$를 찾아낼 수 있으니까. ($\ep$을 반토막 내서 등호 빼버릴 수 있음.) \\ \\
따라서 $\sup \{ \mu (K) : K \subset A-B, K \text{ compact} \} \ge \mu(A-B)$이다.
그리고 $\mu$의 monotone에 의해, $\sup \{ \mu (K) : K \subset A-B, K \text{ compact} \} \le \mu(A-B)$. 
따라서 $\sup \{ \mu (K) : K \subset A-B, K \text{ compact} \} = \mu(A-B)$. $A-B \subset A$에서 finite 
조건은 쉽게 알 수 있다. 따라서, $A-B \in \M_F$. \\ \\
$A \cup B = (A - B) \cup B$, $A - B \in \M_F$, 따라서 $A \cup B \in \M_F$. 
$A \cap B = A - (A - B)$, $A - B \in \M_F$, 따라서 $A \cup B \in \M_F$. 

\paragraph{STEP VII}

$\M$ is a $\sigma$-algebra in $X$ which contains all Borel sets.

\subheading{proof}

1. $A \in \M$이면, $A^c \in \M$임을 보이자. \\
Let $K$ be an arbitrary compact set in $X$.
$A \in \M$, then $A^c \cap K$ = $K - (A \cap K)$, 이 때 compact set $K \in \M_F$이고, $\M$의 정의에 의해
$A \cap K \in \M_F$, 따라서 $A^c \cap K \in \M_F$, for arbitrary compact set $K$. 따라서 $A^c \in \M$. \\ \\

2. $\M$이 countable union에 대해 닫혀있음을 보이자. \\
Let $K$ be an arbitrary compact set in $X$.
$A_i \in \M$인 $A_i$들에 대해, $A = \bigcup _1^\infty A_i$라 하자. $B_1 = A_1 \cap K$라 두고, $B_n$을 다음과
같이 정의하자.
\begin{equation}
  B_n = (A_n \cap K) - (B_1 \cup \cdots \cup B_{n-1}) \qquad (n = 2, 3, 4, \dots).
\end{equation}
즉, $B_n$은 $K$ 중에서 $A_1$부터 $A_{n-1}$을 모두 피하다가 $A_n$과 딱 겹치는 부분을 의미한다. 이렇게 하면, 
$B_n$을 만들 때 $B_{n-1}$까지를 다 빼버리니까 $\{B_n\}$이 disjoint임은 자명하다. 그리고 $A_1 \in \M_F$, 
$K \in \M_F$(since $K$ is compact), 따라서 $A_1 \cap K \in \M_F$(by STEP VI), 그리고 $B_1$부터 $B_{n-1}$까지 
$\M_F$에 속한다고 가정하면 $B_1 \cup \cdots \cup B_{n-1} \in \M_F$(by STEP VI), 그리고 $A_n \in \M_F$,
따라서 $B_n \in \M_F$, induction에 의해 $B_n \in \M_F$ for all $n$. 이 때 $A_n \cap K = \bigcup _1^n B_i$, 
$A \cap K = \bigcup _1^\infty B_n$(마찬가지로 엄밀하게는 by induction)에서 $\bigcup _1^\infty B_n \in 
\M_F$(By STEP VI), $A \cap K \in \M_F$, for arbitrary compact set $K$. 따라서 $A \in \M$. \\ \\

3. 전체집합 $X \in \M$임을 보이자. \\
모든 closed set $C$는 $\M$에 속한다. 왜냐하면, 임의의 compact set $K$에 대해, $C \cap K$는 compact이고, 
compact set은 항상 $\M_F$에 속하기 때문이다. $X$는 closed이므로, $X \in \M$. \\ \\

모든 closed set이 $\M$에 속한다는 것은, 모든 open set이 $\M$에 속한다는 말과 같다. 왜냐하면, 어떤 open set $V$에 
대해, $V^c$는 closed이므로 $V^c \in \M$이니까 $(V^c)^c = V \in \M$. 따라서, $\M$은 모든 open set을 포함하는 
$\sigma$-algebra이므로, 모든 Borel set을 포함해야 한다. (Borel set이 제일 작은거니까.)

\paragraph{STEP VIII}

$\M_F$ consists of precisely those sets $E \in \M$ for which $\mu (E) < \infty$.

$\M_F$는 사실 $\M$ 중에서 $\mu(E)<\infty$인 $E$들만 모아놓았던 것이었다. 이 STEP을 통해, Theorem의 (d)번을 
증명할 수 있다.

\subheading{proof}

($\Rightarrow$) If $E \in \M_F$, then STEP II에서 $K \in \M_F$, STEP VI에 의해 $E \cap K \in \M_F$ for every compact 
$K$이다. \\ \\
($\Leftarrow$) Suppose $E \in \M$ and $\mu(E) < \infty$. (2)번 식, $\mu$의 정의에 의해  $\exists$ open set 
$V \supset E$ with $\mu(V) < \mu(E) + \ep$ for any $\ep > 0$. STEP III에 의해, such $V$ is in $\M_F$. STEP V에 의해,
$\exists$ open set $W$ and $\exists$ compact set $K$ such that $K \subset V \subset W$ and $\le \mu(W - K) < \ep$. 
$E \in \M$이고, $\M$의 정의에 의해 $E \cap K \in \M_F$이므로, (3)번 식에 의해
$\exists$ compact set $H \subset E \cap K$ with 
\begin{equation*}
  \mu (E \cap K) - \ep < \mu(H).
\end{equation*}
Since $E = (E \cap K) \cup (E \cap K^c) \subset (E \cap K) \cup (W - K)$, it follows that
\begin{equation*}
  \mu (E) \le \mu (E \cap K) + \mu(W - K) < \mu (H) + 2 \ep.
\end{equation*}
즉, $\mu (E) \le \mu (H)$ for all compact set $H \subset E$ $\Rightarrow$ $\mu (E) \le \sup \{ \mu(H) : H \subset E, 
H \text{ compact} \}$, $\mu$의 monotone 성질에 의해 $\mu (E) \ge \sup \{ \mu(H) : H \subset E, H \text{ compact} \}$,
따라서 $\mu (E) = \sup \{ \mu(H) : H \subset E, H \text{ compact} \}$, $E \in \M_F$.

\paragraph{STEP IX}

$\mu$ is a measure on $\M$.

\subheading{proof}

만약 $\mu$의 값에 $\infty$가 있는 경우, 양쪽 모두 $\infty$가 됨. (한쪽은 monotone, 한쪽은 0이상 더하기에 무한대 있어서.)
모두 finite 한 경우, STEP IV와 STEP VIII에 의해 바로 증명됨.

\paragraph{STEP X}

For every $f \in C_c(X)$, $\Lambda f = \int_x f d\mu$.

\subheading{proof}

complex function $f = u + iv$에 대해, \begin{equation*}
  \int_E f d\mu = \int_E u^+ d \mu - \int_E u^- d \mu + i \int_E v^+ d \mu - i \int_E v^- d \mu
\end{equation*}
이고, $u^+, u^-, v^+, v^-$는 모두 0 이상인 real function이므로, real function에 대해서만 (a)를 증명하면
$\Lambda$의 linearlity에 의해 complex function에 대해서도 보일 수 있다. 
그리고 우리는 \begin{equation}
  \Lambda f \le \int_X f d \mu
\end{equation}
for every real $f \in C_c(X)$만 증명하면 된다. 왜냐면, (16)번 식으로 인해
\begin{equation*}
  - \Lambda f = \Lambda (-f) \le \int_X(-f) d \mu = - \int_X f d \mu
\end{equation*}
가 성립하기 때문이다. \\ \\
Real function $f \in C_c(X)$에 대해, $f$의 support를 $K$라고 하자. Theorem 2.10의 Corollary에 의해, 
$f(K)$의 range는 complex plane에서의 compact subset이다. $f$는 real function이므로, $f(X) \in [a, b]$
인 $a, b$를 찾을 수 있다. 그리고 임의의 $\ep > 0$에 대해, $a$와 $b$ 사이에 간격이 $\ep$보다 작아지도록 $y_i$를
다음과 같이 배치하자. \begin{equation}
  y_0 < a < y_1 < \cdots < y_n = b.
\end{equation}
구간의 길이가 finite이므로, finite개의 $y_n$을 찾아낼 수 있다. $E_i$를 \begin{equation}
  E_i = \{ x: y_{i-1} < f(x) \le y_i \} \cap K \qquad (i = 1, \dots, n).
\end{equation}
으로 정의하자. 이는 $f$의 range를 $\ep$보다 작게 조각내놓은 것과 대응되는 preimage 조각들이다. 이들은
disjoint하면서 합집합이 $K$가 된다. \\ \\
$f \in C_c (X)$에서 $f$가 continuous이므로 $f$는 Borel measurable이다. 그래서 $(y_{i-1}, y_i]$의
preimage인 $E_i$들은 $X$에서 Borel set이다. (왜냐하면 $(y_{i-1}, y_i]$은 Borel set in $\R$, 
see Theorem 1.12 (b).) \\ \\
$E_i \subset K$에서, $\mu (K) < \infty$이므로 식 (2)에 의해 \begin{equation}
  \mu(W_i) < \mu(E_i) + \frac{\ep}{n} \qquad (i = 1, \dots, n)
\end{equation}
인 $W_i \supset E_i$를 $E_i$마다 찾아낼 수 있다. 그리고 $W'_i = f^{-1}([- \infty, y_i + \ep))$으로 정의하면, 
$W'_i$는 open이고 $W'_i \supset E_i$. 그리고 $V_i = W_i \cap W'_i$로 두면, $V_i \supset E_i$는 (19)번 
식을 만족하면서 $f(x) < y_i + \ep$ for all $x \in V_i$ 인 open set. 따라서 Theorem 2.13에 의해, $h_i \prec V_i$
such that $\sum h_i = 1$ on $K$인 $h_i$들을 찾을 수 있다. 이 때, $\sum h_i = 1 \Rightarrow f = \sum h_i f$이고,
$K \prec \sum h_i$이므로 STEP II에 의해 \begin{equation*}
  \mu(K) \le \Lambda (\sum h_i) = \sum \Lambda h_i
\end{equation*}
QQQQQ \\
 Q. $K \prec \sum h_i$ 맞음? Theorem에서는 $\sum h_i = 1$ on $K$라고만 했는데, $K$ 밖에서 $h_i$들 중 일부가 만나 
 1보다 커지면 어떡하지? 2.13 증명과정 중 식 (3)에서 $\sum h_i \le 1$임을 확인할 수 있음. \\
QQQQQ \\
따라서 $h_i f \le (y_i + \ep) h_i$ on $E_i \subset V_i$, $y_i - \ep < f(x)$ on $E_i \subset V_i$이고, 
$E_i$들은 disjoint하고 합집합이 $K$임으로, 다음과 같이 쓸 수 있다.
\begin{align*}
  \Lambda f &= \Lambda(\sum_{i=1}^{n} h_i f) = \sum_{i=1}^{n} \Lambda(h_i f) \le \sum_{i=1}^{n} \Lambda (y_i + \epsilon)h_i \\
  &= \sum_{i=1}^{n} (y_i + \epsilon)\Lambda h_i = \sum_{i=1}^{n} (|a| + y_i + \epsilon)\Lambda h_i - |a| \sum_{i=1}^{n} \Lambda h_i \\
  &\le \sum_{i=1}^{n} (|a| + y_i + \epsilon)[\mu(E_i) + \epsilon/n] - |a| \mu(K) \\
  &= \frac{\ep}{n} \sum_{i=1}^{n} (|a| + y_i + \epsilon) + \sum_{i=1}^{n} (|a| + y_i + \epsilon)\mu(E_i) - |a| \mu(K) \\
  &= \sum_{i=1}^{n} \[ (y_i - \epsilon)\mu(E_i) + 2 \ep \mu(E_i) + |a| \mu (E_i) \] - |a| \mu(K) + \frac{\ep}{n} \sum_{i=1}^{n} (|a| + y_i + \epsilon)\\
  &= \sum_{i=1}^{n} (y_i - \epsilon)\mu(E_i) + 2\epsilon \mu(K) + \frac{\epsilon}{n} \sum_{i=1}^{n} (|a| + y_i + \epsilon) \\
  &\le \int_X f \, d\mu + \epsilon[2\mu(K) + |a| + b + \epsilon].
\end{align*}
(Since 1. $h_i \prec V_i$, $\Lambda h_i \le \mu(V_i)$ by the definition of $\mu (V_i)$ $\Rightarrow$ 
$\Lambda h_i \le \mu(V_i) < \mu (E_i) + \ep/n$) \\ \\
(2. $\mu(K) \le \sum_{i=1}^{n} \Lambda h_i$는 아까 했음.) \\ \\
(3. $\sum_{i=1}^{n} \mu(E_i)  = \mu(K)$, by STEP IV, and $E_i$ are Borel sets in $X$, hence $E_i \in \M$.) \\ \\
(4. $\sum_{i=1}^{n} (y_i - \epsilon)\mu(E_i) \le \int_X f \, d\mu$, since $y_i - < \le y_{i-1} < f(x)$ for all $x \in E_i$) \\ \\
(5. $\sum_{i=1}^{n} y_i < \sum_{i=1}^{n} b = nb$)

\end{document}

