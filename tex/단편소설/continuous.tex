\documentclass[11pt]{article}
\usepackage{amsmath}
\usepackage{amssymb}
\usepackage{amsthm}
\usepackage{enumerate}
\usepackage[notcite,notref]{showkeys}
\usepackage{kotex}

\usepackage{url}
\usepackage{hyperref}

\usepackage{graphicx}

% PAGE SETTING ---------------------------------------

\textheight 22.5  true cm
\textwidth 15 true cm
\voffset -1.0 true cm
\hoffset -1.0 true cm
\marginparwidth= 2 true cm
\renewcommand{\baselinestretch}{1.8}

% CUSTOM COMMAND -------------------------------------------

\renewcommand{\(}{\left(}
\renewcommand{\)}{\right)}
\renewcommand{\[}{\left[}
\renewcommand{\]}{\right]}

\newcommand{\inp}[2]{\langle #1,#2 \rangle}
\newcommand{\diag}{{\rm diag}}

\newcommand{\dist}{\text{dist }}
\newcommand{\supp}{\text{supp }}

\newcommand{\R}{\mathbb{R}}
\newcommand{\Rp}{\mathbb{R}_+}
\newcommand{\Rpp}{\mathbb{R}_{++}}
\newcommand{\C}{\mathbb{C}}
\newcommand{\N}{\mathbb{N}}
\newcommand{\Z}{\mathbb{Z}}
\newcommand{\Q}{\mathbb{Q}}

\newcommand{\ep}{\epsilon}
\newcommand{\pa}{\partial}

\newcommand{\mcC}{\mathcal{C}}
\newcommand{\mcH}{\mathcal{H}}
\newcommand{\mcT}{\mathcal{T}}
\newcommand{\mcV}{\mathcal{V}}
\newcommand{\mcG}{\mathcal{G}}
\newcommand{\mcE}{\mathcal{E}}
\newcommand{\mcW}{\mathcal{W}}

\newcommand{\suchthat}{ \; \big| \; }

%-----------------------------------------------------------------

\newcommand{\tags}[1]{#1}
\newenvironment{text-box}{}{}
\newcommand{\subheading}[1]{#1}

% ----------------------------------------------------------------

\begin{document}
\title{Continuous, Lipschitz continuous, Hölder continuous}
\date{2025-05-23}
\tags{}
\author{skwodnjs}

\maketitle

\section{Continuous}

\subsection{엡실론 델타 논법}

Let $X, Y$ be metric spaces, $f: X \to Y$ be a function. \\
$f$ is continuous at $x \in X$ if \\
\begin{text-box}
    $\forall \epsilon > 0$, $\exists \delta > 0$ such that
    \begin{equation}
        d_X(x, y) < \delta \Rightarrow d_Y(f(x), f(y)) < \epsilon
    \end{equation}
\end{text-box}

\subsection{Topological Space}

Let $X, Y$ be topological spaces, $f: X \to Y$ be a function. \\
$f$ is continuous at $x \in X$ if \\
for every neighborhood $V$ of $f(x)$, there exists a neighborhood $U$ of $x$ 
such that $f(U) \subset V$.

$f(U) \subset V$를 풀어 쓰면 $x \in U \Rightarrow f(x) \in V$이다.

이 정의는 거리공간에서 정의했던 연속에 대한 정의를 위상공간으로 확장한 정의이다. 
거리공간에서의 정의를 완벽하게 유지하면서도, 위상공간에서의 연속성에 대해 논하고 있다. 
$X, Y$가 거리공간일 때, 위 정의는 엡실론 델타 논법과 동치이다.

\subheading{proof}

\begin{text-box}
    Let $X, Y$ be metric spaces, $f: X \to Y$ be a function. \\
    (i) Assume that for every neighborhood $V$ of $f(x)$, there exists a neighborhood $U$ of $x$ 
    such that $f(U) \subset V$. \\
    Then $\forall \epsilon > 0$, $\{ y \in Y \suchthat d_Y(f(x), y) < \epsilon \}$ 
    is a neighborhood of $f(x)$, and by the assumption, there exists a neighborhood $U$ of $x$ 
    such that $f(U) \subset \{ y \in Y \suchthat d_Y(f(x), y) < \epsilon \}$. And we can find 
    $\delta$ such that $\{ z \in X \suchthat d_X(x, z) < \delta \} \subset U$, since $U$ is open.
       
    (ii) Assume that $\forall \epsilon > 0$, $\exists \delta > 0$ such that
    \begin{equation}
        d_X(x, y) < \delta \Rightarrow d_Y(f(x), f(y)) < \epsilon
    \end{equation}
    Let $V$ be a neighborhood of $f(x)$. And Let $E$ be a preimage of $V$. That is, $E := f^{-1} (V)$. 
    Then, for each $x \in E$, we can find $\epsilon_x > 0$ such that $\mathcal B_Y(f(x), \epsilon_x) \subset V$, since $V$ is open. 
    And by assumption, we can find $\delta_x > 0$ such that $f(\mathcal B_X(x, \delta_x)) \subset \mathcal B_Y(f(x), \epsilon_x)$. 
    Define $\displaystyle U := \bigcup_{x \in E} \mathcal B_X(x, \delta_x)$. Then \begin{equation}
        f(U) = f(\bigcup_{x \in E} \mathcal B_X(x, \delta_x)) = \bigcup_{x \in E} f(\mathcal B_X(x, \delta_x)) \subset V
    \end{equation}
\end{text-box}

\subsection{Continuous on X}

연속의 정의를 다음과 같이 확장할 수 있다. \\
\begin{text-box}
    Let $X, Y$ be topological spaces, $f: X \to Y$ be a function. \\
    $f$ is continuous on $X$ if for every open set $V$ in $Y$, $f^{-1}(V)$ is open in $X$. \\
    $f$ is continuous on $E \subseteq X$ if for every open set $V$ in $Y$, $f^{-1}(V)$ is open in the subspace topology on $E$.
\end{text-box}

위와 같은 정의는 아래와 동치이다. \\
\begin{text-box}
    Let $X, Y$ be topological spaces, $f: X \to Y$ be a function. \\
    $f$ is continuous on $X$ if $f$ is continuous at $\forall x \in X$. \\
    $f$ is continuous on $E \subseteq X$ if $f$ is continuous at $\forall x \in E$.
\end{text-box}

\subheading{proof}

"추가 바람".

\section{Lipschitz Continuous}

어떤 점을 중심으로 f의 제한이 생기면 그 제한 안으로 골인시키는 x가 존재해야 그 점에서 continuous라고 하는데, 
그 골인시키는 x의 구간이 f의 제한에 대하여 linear하면 그게바로 lipschitz.

\end{document}