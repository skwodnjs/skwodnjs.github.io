% ---------- This is the blog post preset. -----------

\documentclass[11pt,reqno]{article}

% PACKAGE --------------------------------------------

\usepackage{amsmath}
\usepackage{amssymb}
\usepackage{amsthm}

\usepackage{enumerate}
\usepackage[notcite,notref]{showkeys}
\usepackage[usenames]{color}
\usepackage{url}
\usepackage{hyperref}
\usepackage{kotex}

\usepackage{graphicx}

% PAGE SETTING ---------------------------------------

\textheight 22.5  true cm
\textwidth 15 true cm
\voffset -1.0 true cm
\hoffset -1.0 true cm
\marginparwidth= 2 true cm
\renewcommand{\baselinestretch}{1.2}

% blog post setting
\usepackage{titlesec}
\setlength{\parindent}{0pt}
\titleformat{\section}[block]{\normalfont\Large\bfseries}{}{0pt}{}
\titleformat{\subsection}[block]{\normalfont\large\bfseries}{}{0pt}{}
\titlespacing*{\subsection}{0pt}{3.0ex plus 1ex minus .2ex}{2.0ex plus .2ex}
\titleformat{\subsubsection}[block]{\normalfont}{}{0pt}{}
\titlespacing*{\subsubsection}{0pt}{3.0ex plus 1ex minus .2ex}{2.0ex plus .2ex}
% blog post setting end

% COSTOM COMMAND -------------------------------------

\renewcommand{\(}{\left(}
\renewcommand{\)}{\right)}
\renewcommand{\[}{\left[}
\renewcommand{\]}{\right]}

\newcommand{\inp}[2]{\langle #1,#2 \rangle}

\newcommand{\diag}{{\rm diag}}
\newcommand{\supp}{\text{supp }}

\newcommand{\R}{\mathbb{R}}
\newcommand{\Rp}{\mathbb{R}_+}
\newcommand{\Rpp}{\mathbb{R}_{++}}
\newcommand{\C}{\mathbb{C}}
\newcommand{\N}{\mathbb{N}}
\newcommand{\Z}{\mathbb{Z}}
\newcommand{\Q}{\mathbb{Q}}

\newcommand{\ep}{\epsilon}
\newcommand{\pa}{\partial}

\newcommand{\mcC}{\mathcal{C}}
\newcommand{\mcH}{\mathcal{H}}
\newcommand{\mcT}{\mathcal{T}}
\newcommand{\mcV}{\mathcal{V}}
\newcommand{\mcG}{\mathcal{G}}
\newcommand{\mcE}{\mathcal{E}}
\newcommand{\mcW}{\mathcal{W}}

\newcommand{\st}{ \; \big| \; }

% blog post setting
\usepackage{tcolorbox}
\newenvironment{note}
  {\begin{tcolorbox}[
    colback=white, 
    colframe=gray!50, 
    boxrule=0.5pt,
    fontupper=\normalfont
  ]}
  {\end{tcolorbox}}
\newcommand{\subheading}[1]{\vspace{1em}{\noindent\large\bfseries \textlangle{} #1 \textrangle{} \par}\vspace{1em}}
\newcommand{\gap}{\vspace{5em}}
% blog post setting end

% COSTOM COMMAND -------------------------------------

\newcommand{\M}{\mathfrak{M}}

% DOCUMENT -------------------------------------------

\begin{document}

\title{3.1 Definition}
\author{Jeawon Na}
\date{}

\maketitle

\section{3.1 Definition}

A real function $\varphi$ defined on a segment $(a, b)$, where $-\infty \le a < b \le \infty$, is called \textit{convex}
if the inequality
\begin{equation} \label{eq:convex}
  \varphi ((1-\lambda) x + \lambda y) \le (1 - \lambda) \varphi(x) + \lambda \varphi(y)
\end{equation}
holds whenever $a < x < b$, $a < y < b$, and $0 \le \lambda \le 1$.

\gap

식 (\ref{eq:convex})은 다음 식과 동치이다.
\begin{equation}
  \frac{\varphi(t) - \varphi(s)}{t - s} \le \frac{\varphi(u) - \varphi(t)}{u - t}
\end{equation}
whenever $a < s < t < u < b$. [1\label{note:1}]

\gap

같은 방식으로, 다음 식과도 동치이다.
\begin{equation}
  \frac{\varphi(t) - \varphi(s)}{t - s} \le \frac{\varphi(u) - \varphi(s)}{u - s}
\end{equation}
whenever $a < s < t < u < b$.

\gap

같은 방식으로, 다음 식과도 동치이다.
\begin{equation}
  \frac{\varphi(u) - \varphi(s)}{u - s} \le \frac{\varphi(u) - \varphi(t)}{u - t}
\end{equation}
whenever $a < s < t < u < b$.

\gap

만약 $\phi$가 $(a, b)$에서 미분 가능한 함수였다면, mean value theorem에 의해, 식 (\ref{eq:convex})은 다음과도 동치이다.
\begin{equation}
  \phi'(s) \le \phi'(t)
\end{equation}
whenever $a < s < t < b$. 즉, 미분한 함수가 증가함수이면 convex이다.

\begin{note}
  note \ref{note:1}

  Let $t = (1-\lambda) s + \lambda u$. Then
  \begin{equation}
    \lambda = \frac{t-s}{u-s}, \quad 1 - \lambda = \frac{u-t}{u-s}.
  \end{equation}
  식 (\ref{eq:convex})에 대입하면
  \begin{align}
    \varphi (t) & \le \frac{u-t}{u-s} \varphi (s) + \frac{t-s}{u-s} \varphi (u), \\
    (u - s) \varphi (t) & \le (u - t) \varphi (s) + (t - s) \varphi (u), \\
    (u - t + t - s) \varphi (t) & \le (u - t) \varphi (s) + (t - s) \varphi (u), \\
    (u - t)(\varphi (t) - \varphi (s)) & \le (t - s) (\varphi (u) - \varphi (t)), \\
    \therefore \frac{\varphi(t) - \varphi(s)}{t - s} & \le \frac{\varphi(u) - \varphi(t)}{u - t}
  \end{align}
  
  역방향도 그대로 성립. 동치임을 알 수 있음.
\end{note}

\end{document}