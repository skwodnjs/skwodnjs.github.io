% ---------- This is the blog post preset. -----------

\documentclass[11pt,reqno]{article}

% PACKAGE --------------------------------------------

\usepackage{amsmath}
\usepackage{amssymb}
\usepackage{amsthm}

\usepackage{enumerate}
\usepackage[notcite,notref]{showkeys}
\usepackage[usenames]{color}
\usepackage{url}
\usepackage{hyperref}
\usepackage{kotex}

\usepackage{graphicx}

% PAGE SETTING ---------------------------------------

\textheight 22.5  true cm
\textwidth 15 true cm
\voffset -1.0 true cm
\hoffset -1.0 true cm
\marginparwidth= 2 true cm
\renewcommand{\baselinestretch}{1.2}

% blog post setting
\usepackage{titlesec}
\setlength{\parindent}{0pt}
\titleformat{\section}[block]{\normalfont\Large\bfseries}{}{0pt}{}
\titleformat{\subsection}[block]{\normalfont\large\bfseries}{}{0pt}{}
\titlespacing*{\subsection}{0pt}{3.0ex plus 1ex minus .2ex}{2.0ex plus .2ex}
\titleformat{\subsubsection}[block]{\normalfont}{}{0pt}{}
\titlespacing*{\subsubsection}{0pt}{3.0ex plus 1ex minus .2ex}{2.0ex plus .2ex}
% blog post setting end

% COSTOM COMMAND -------------------------------------

\renewcommand{\(}{\left(}
\renewcommand{\)}{\right)}
\renewcommand{\[}{\left[}
\renewcommand{\]}{\right]}

\newcommand{\inp}[2]{\langle #1,#2 \rangle}

\newcommand{\diag}{{\rm diag}}
\newcommand{\supp}{\text{supp }}

\newcommand{\R}{\mathbb{R}}
\newcommand{\Rp}{\mathbb{R}_+}
\newcommand{\Rpp}{\mathbb{R}_{++}}
\newcommand{\C}{\mathbb{C}}
\newcommand{\N}{\mathbb{N}}
\newcommand{\Z}{\mathbb{Z}}
\newcommand{\Q}{\mathbb{Q}}

\newcommand{\ep}{\epsilon}
\newcommand{\pa}{\partial}

\newcommand{\mcC}{\mathcal{C}}
\newcommand{\mcH}{\mathcal{H}}
\newcommand{\mcT}{\mathcal{T}}
\newcommand{\mcV}{\mathcal{V}}
\newcommand{\mcG}{\mathcal{G}}
\newcommand{\mcE}{\mathcal{E}}
\newcommand{\mcW}{\mathcal{W}}

\newcommand{\st}{ \; \big| \; }

% blog post setting
\usepackage{tcolorbox}
\newenvironment{textbox}
  {\begin{tcolorbox}[
    colback=gray!10, 
    colframe=gray!50, 
    boxrule=0.5pt,
    fontupper=\normalfont
  ]}
  {\end{tcolorbox}}
\newcommand{\subheading}[1]{\vspace{1em}{\noindent\large\bfseries \textlangle{} #1 \textrangle{} \par}\vspace{1em}}
% blog post setting end

% COSTOM COMMAND -------------------------------------

\newcommand{\M}{\mathfrak{M}}

% DOCUMENT -------------------------------------------

\begin{document}

\title{CHAPTER THREE : $L^p$- SPACES}
\author{Jeawon Na}
\date{}

\maketitle

Walter Rudin - Real and Complex Analysis-McGraw-Hill Education (1986)

\section{Convex Functions and Inequalities}

\subsection{3.1 Definition}

A real function $\phi$ defined on a segment $(a, b)$, where $-\infty \le a < b \le \infty$, is called \textit{convex}
if the inequality
\begin{equation}
  \phi ((1-\lambda) x + \lambda y) \le (1 - \lambda) \phi(x) + \lambda \phi(y)
\end{equation}
holds whenever $a < x < b$, $a < y < b$, and $0 \le \lambda \le 1$.

\subsection{3.2 Theorem}

If $\varphi$ is convex on $(a, b)$ then $\varphi$ is continuous on $(a, b)$.

$\varphi$가 open set 위에서 convex일 때에만 continuous가 보장된다. 예를 들면, $\varphi(x) = 0$ for $x \in (0, 1)$ and 
$\varphi(x) = 1$ for $x = 1$인 함수 $\varphi$는 $(0, 1]$에서 convex이지만 continuous는 아니다.

\subsection{3.3 Theorem (Jensen's Inequality)}

Let $\mu$ be a positive measure on a $\sigma$-algebra $\M$ in a set $\Omega$, so that $\mu(\Omega) = 1$. If $f$ is
a real function in $L^1 (\mu)$, if $a < f(x) < b$ for all $x \in \Omega$, and if $\varphi$ is convex on $(a, b)$, then
\begin{equation}
  \varphi \( \int_\Omega f d\mu \) \le \int_\Omega (\phi \circ f) d\mu.
\end{equation}

\subsection{3.4 Definition}

If $p$ and $q$ are positive real numbers such that $p + q = pq$, or equivalently 
\begin{equation} \label{eq:conjugate exponents}
  \frac{1}{p} + \frac{1}{q} = 1,
\end{equation}
then we call $p$ and $q$ a pair of \textit{conjugate exponents}. It is clear that (\ref{eq:conjugate exponents})
implies $1 < p < \infty$ and $1 < q < \infty$. An important special case is $p = q = 2$. 

As $p \to 1$, (\ref{eq:conjugate exponents}) forces $q \to \infty$. Consequently 1 and $\infty$ are also regarded as a pair of 
conjugate exponents. Many analysis denote th exponents conjugate to $p$ by $p'$, often without saying so explicitly.

\subsection{3.5 Theorem}

Let $p$ and $q$ be conjugate exponents, $1 < p < \infty$. Let $X$ be a measure space, with measure $mu$. Let $f$ and $g$ be 
measurable functions on $X$, with range in $[0, \infty]$. Then
\begin{equation} \label{Holder's}
  \int_X fg d \mu \le \left\{ \int_X f^p \right\}^{1/p} \left\{ \int_X g^q \right\}^{1/q}
\end{equation}
and
\begin{equation} \label{Minkowski's}
  \left\{ \int_X (f + g)^p d \mu \right\}^{1/p} \le \left\{ \int_X f^p \right\}^{1/p} +  \left\{ \int_X g^p \right\}^{1/p}
\end{equation}.
The inequality (\ref{Holder's}) is Holder's; (\ref{Minkowski's}) is Minkowski's. If $p=q=2$, (\ref{Holder's}) is known as
the Schwarz inequality.



\end{document}