% ---------- This is the blog post preset. -----------

\documentclass[11pt,reqno]{article}

% PACKAGE --------------------------------------------

\usepackage{amsmath}
\usepackage{amssymb}
\usepackage{amsthm}

\usepackage{enumerate}
\usepackage[notcite,notref]{showkeys}
\usepackage[usenames]{color}
\usepackage{url}
\usepackage{hyperref}
\usepackage{kotex}

\usepackage{graphicx}

% PAGE SETTING ---------------------------------------

\textheight 22.5  true cm
\textwidth 15 true cm
\voffset -1.0 true cm
\hoffset -1.0 true cm
\marginparwidth= 2 true cm
\renewcommand{\baselinestretch}{1.2}

% blog post setting
\usepackage{titlesec}
\setlength{\parindent}{0pt}
\titleformat{\section}[block]{\normalfont\Large\bfseries}{}{0pt}{}
\titleformat{\subsection}[block]{\normalfont\large\bfseries}{}{0pt}{}
\titlespacing*{\subsection}{0pt}{3.0ex plus 1ex minus .2ex}{2.0ex plus .2ex}
\titleformat{\subsubsection}[block]{\normalfont}{}{0pt}{}
\titlespacing*{\subsubsection}{0pt}{3.0ex plus 1ex minus .2ex}{2.0ex plus .2ex}
% blog post setting end

% COSTOM COMMAND -------------------------------------

\renewcommand{\(}{\left(}
\renewcommand{\)}{\right)}
\renewcommand{\[}{\left[}
\renewcommand{\]}{\right]}

\newcommand{\inp}[2]{\langle #1,#2 \rangle}

\newcommand{\diag}{{\rm diag}}
\newcommand{\supp}{\text{supp }}

\newcommand{\R}{\mathbb{R}}
\newcommand{\Rp}{\mathbb{R}_+}
\newcommand{\Rpp}{\mathbb{R}_{++}}
\newcommand{\C}{\mathbb{C}}
\newcommand{\N}{\mathbb{N}}
\newcommand{\Z}{\mathbb{Z}}
\newcommand{\Q}{\mathbb{Q}}

\newcommand{\ep}{\epsilon}
\newcommand{\pa}{\partial}

\newcommand{\mcC}{\mathcal{C}}
\newcommand{\mcH}{\mathcal{H}}
\newcommand{\mcT}{\mathcal{T}}
\newcommand{\mcV}{\mathcal{V}}
\newcommand{\mcG}{\mathcal{G}}
\newcommand{\mcE}{\mathcal{E}}
\newcommand{\mcW}{\mathcal{W}}

\newcommand{\st}{ \; \big| \; }

% blog post setting
\usepackage{tcolorbox}
\newenvironment{note}
  {\begin{tcolorbox}[
    colback=white, 
    colframe=gray!50, 
    boxrule=0.5pt,
    fontupper=\normalfont
  ]}
  {\end{tcolorbox}}
\newcommand{\subheading}[1]{\vspace{1em}{\noindent\large\bfseries \textlangle{} #1 \textrangle{} \par}\vspace{1em}}
\newcommand{\gap}{\vspace{5em}}
% blog post setting end

% COSTOM COMMAND -------------------------------------

\newcommand{\M}{\mathfrak{M}}

% DOCUMENT -------------------------------------------

\begin{document}

\title{3.3 Theorem (Jensen's Inequality)}
\author{Jeawon Na}
\date{}

\maketitle

\section{3.3 Theorem (Jensen's Inequality)}

Let $\mu$ be a positive measure on a $\sigma$-algebra $\M$ in a set $\Omega$, so that $\mu(\Omega) = 1$. If $f$ is
a real function in $L^1 (\mu)$, if $a < f(x) < b$ for all $x \in \Omega$, and if $\varphi$ is convex on $(a, b)$, then
\begin{equation}
  \varphi \( \int_\Omega f d\mu \) \le \int_\Omega (\phi \circ f) d\mu.
\end{equation}

\gap

convex function $\varphi: (a, b) \to \R$이 있을 때, $\mu$와 $f$는 구간 $(a, b)$의 가중평균을 나타낸다고 볼 수 있다. 
$\frac{1}{\mu(\Omega)} \int_\Omega f d\mu$는 $\Omega$ 위에서 $f$의 평균인데, $\mu (\Omega) = 1$이기 때문이다. 
특히 $f$가 simple function일 때에는, 어떤 $c \in (a, b)$의 가중치를 $f^{-1}({c})$로 볼 수 있다.

\gap

\subsection{증명요약}

$t = \int_\Omega f d\mu$로 놓고, convex 식을 정리하면 된다. 이때, $a < s < b$인 모든 $s$에 대해 성립하는 식을 하나 만들어놓고, 
$s = f(x)$를 대입한 뒤 $\mu$로 적분하면 식이 정리된다.

\gap

\subsection{proof}

Let $t = \int_\Omega f d\mu$. Then $a < t < b$[1\label{note:1}].
Since $\varphi$ is convex,
\begin{equation} \label{eq:convex}
  \frac{\varphi (t) - \varphi (s)}{t-s} \le \frac{\varphi (u) - \varphi (t)}{u-t}
\end{equation}
whenever $a < s < t < u < b$. ($a, t, b$는 fixed point).
Let $\beta = \sup_{s, a<s<t} \frac{\varphi (t) - \varphi (s)}{t-s}$. Then,
\begin{equation} \label{eq:convex1}
  \beta \ge \frac{\varphi (t) - \varphi (s)}{t-s} \Rightarrow \beta (t - s) \ge \varphi (t) - \varphi (s)
\end{equation} for any $s$ with $a<s<t$, by the definition of supremum, and
\begin{equation} \label{eq:convex2}
  \beta \le \frac{\varphi (u) - \varphi (t)}{u-t} \Rightarrow \beta (u - t) \le \varphi (u) - \varphi (t) \Rightarrow \beta (t - u) \ge \varphi (t) - \varphi (u)
\end{equation} for any $u$ with $t < u < b$, by (\ref{eq:convex}).
By (\ref{eq:convex1}) and (\ref{eq:convex2}), $\beta (t - s) \ge \varphi (t) - \varphi (s)$ for all $s$ such that $a < s < b$.
$a < f(x) < b$ for all $x \in \Omega$이므로, put $s = f(x)$.
\begin{align} \label{eq:phi f}
  \beta (t - f(x)) \ge \varphi (t) - \varphi (f(x)) \\
  \varphi (f(x)) \ge \varphi (t) - \beta (t - f(x))
\end{align}
Since $\varphi$ is continuous, $\varphi \circ f$ is measurable.
Integrate both sides of (\ref{eq:phi f}) with respect to $\mu$ gives
\begin{align}
  \int_\Omega \varphi \circ f d\mu &\ge \int_\Omega \varphi(t) d\mu - \beta \(\int_\Omega t d\mu - \int_\Omega f d\mu\) \\
  &= \varphi(t) \cdot \mu(\Omega) - \beta( t \cdot \mu(\Omega) - t) = \int_\Omega \varphi(t) d\mu - \beta(t - t) \label{note:2} \\
  &= \varphi(t) = \varphi \(\int_\Omega f d\mu\).
\end{align}

\begin{note}
  note \ref{note:1}

  $f(x) < g(x)$ for all $x \in X$일 때, $0 < g(x) - f(x)$ for all $x \in X$.
  시발뭐야
  $\therefore \int_X f d\mu < \int_X g d\mu$ if $f < g$ and $\mu(X) > 0$.

\end{note}

\begin{note}
  note \ref{note:2}

  $x$에 대해 적분하는거임. 그리고 $t$는 상수임. 그러니까 상수함수(simple function)로 보고 적분하면 됨.
\end{note}

\end{document}